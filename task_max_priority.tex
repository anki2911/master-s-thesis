\section{Task Maximisation with priority}\label{tmwp}
\noindent
Let us consider an automobile system where we consider a scenario associated with car crash avoidance. In such a system,
camera sensors emit messages, but this messages may not be of much importance to us in the normal scenario. The camera 
sensor messages are of higher criticality because it is mainly responsible in order to maintain the safety of our system.
To avoid a car crash avoidance scenario, the camera sensors can suddenly emit messages more often and may create an overload 
in the system. In such a scenario, the camera sensor messages are to be given much more importance.
\newline
\newline
From the above example, we can say that a priority is a relative concept. Let us consider we have two or more applications 
of different criticality levels executing together on a platform. For safety of the system, a high criticality task should 
not miss its deadline. Therefore, in our system, a task having higher criticality is considered to be of higher priority.
Again between two tasks of same criticality level, one task may have higher priority than another task based on deadline or 
some other task parameters. So priority between two tasks is not a fixed parameter and cannot be assigned during static 
analysis.
\subsection{Problem Formulation}\label{pftp}
\begin{problem}
 Given a set of tasks with different budgets of execution at each criticality level with different priorities. We have 
 considered that a task having larger number of execution modes is having higher priority. Given a memory with rank of size 
 m, we need to ensure that no high priority tasks will miss their deadlines at the cost of any low priority tasks and maximum
 number of tasks gets executed.
\end{problem}
Table \ref{tab3} shows a set of tasks with different criticality levels, different budget for execution at each 
criticality level and priorities assigned to each task. We have assumed that the priorities are assigned in ascending order of 
magnitude. That is, a task having lower priority will have lower value than a task having higher priority.


\begin{table}[t]
 \begin{tabular}{|c|c|c|c|c|c|}\hline
 Task ID & Criticality Levels & Parallel Rank Access  & Percentage of task executed & Deadline & Priority\\ \hline
 1 & {$L_{11}$, $L_{12}$, $L_{13}$} & $L_{11}$ - 6 & $L_{11}$ - 25\% & 4 cycles & 3 \\
   &  & $L_{12}$ - 10 & $L_{12}$ - 35\% & &  \\
   &  & $L_{13}$ - 0 & $L_{13}$ - 0\% & & \\ \hline
 2 & {$L_{21}$, $L_{22}$, $L_{23}$} & $L_{21}$ - 5 & $L_{21}$ - 20\% & 5 cycles & 3\\
   &  & $L_{22}$ - 10 & $L_{22}$ - 30\% & & \\
   &  & $L_{23}$ - 0 & $L_{23}$ - 0\% & & \\ \hline
 3 & {$L_{31}$, $L_{32}$, $L_{33}$, $L_{34}$} & $L_{31}$ - 0 & $L_{31}$ - 0\% & 6 cycles & 4 \\
   &  & $L_{32}$ - 4 & $L_{32}$ - 15\% & & \\
   &  & $L_{33}$ - 8 & $L_{33}$ - 25\% & & \\
   &  & $L_{34}$ - 12 & $L_{34}$ - 40\% & & \\ \hline 
 \end{tabular}
\caption{Task Parameters with task priority}
\label{tab3}
\end{table}
Some conditions that are needed to be satisfied to answer the above maximisation problem are - 
\newline
\newline
Condition 1: In every cycle a task will execute in exactly one of its criticality level.
\newline
\newline
Condition 2: A task must complete 100\% of its execution before its deadline.
\newline
\newline
Condition 3: At any cycle the total number of ranks accessed by all the executing tasks must be less than or equal to the 
number of ranks available to the system.
\newline
\newline
Condition 4: A task may execute completely with priority less than or equal to the highest priority of a task in the system.
\newline
\subsection{Hardness Characterisation of the problem}\label{hcp}

\subsection{Constraint Formulation}\label{cfp}
We have introduced three decision variables $C_{i}$, $S_{i}$ and $K_{i}$.
The decison variable $C_{i}$ is to keep track of the task completion. $C_{i}$ marks the completion of task $T_{i}$.
\begin{align*}
 C_{i} &=
 \begin{cases}
  1	& \text{$i^{th}$ task $T_{i}$ has completed 100\% of the task} \\
  0	& \text{otherwise}
 \end{cases}
\end{align*}
\newline
\newline
The decision variable $S_{i}$ is set when $i^{th}$ task is of highest priority. 
\begin{align*}
 S_{i} &=
 \begin{cases}
  1	& \text{$i^{th}$ task $T_{i}$ has completed its execution and $T_{i}$ is of highest priority0}\\
  0 	& \text{otherwise}
 \end{cases}
\end{align*}
%\newline
We assign some high weights, say $\alpha$, to the tasks having maximum priority and some much lower weights, say $\beta$ to 
the tasks of lower priority. On executing a task of highest priority, a weight $\alpha$ gets added to 
our objective function and on executing a task of lower priority, a weight $\beta$ gets added to our objective function. 
\newline
The values of $\alpha$ and $\beta$ are selected based on the priorities among the different tasks in the system. Generally, 
$\alpha$ $>>$ $\beta$. Here we have chosen $\alpha$ to be 1000 and $\beta$ to be 1.
\newline
\newline
1. Formulating the above conditions into constraints using ILP, we get -
\newline
\begin{center}
 Maximize $\alpha$ * $S_{i}$ + $\beta$ * $C_{i}$ over all tasks, or more specifically, 1000 * $S_{i}$ + $C_{i}$ over all tasks,
 subject to
 \end{center}
\begin{equation}\label{eq1}
\sum_{i = 1}^{n}\sum_{j = 1}^{|L_{i}|} f_{ijk} \leq 1, \forall k \in (1,D_{i})
\end{equation}
\newline
\begin{equation}\label{eq2}
\sum_{k = 1}^{D_{i}} \sum_{j = 1}^{|L_{i}|} E_{ij}(L_{ij}) * f_{ijk} \geq 100 * C_i, \forall i \in (1,n)
\end{equation}
\newline
\begin{equation}\label{eq3}
\sum_{i = 1}^{n} \sum_{j = 1}^{|L_{i}|} R_{ij}(L_{ij}) * f_{ijk} \leq z, \forall k \in (1,D_{i})
\end{equation}
\newline
\begin{equation}
P_{i} * C_{i} \geq P * S_{i}, \forall i \in (1,n)
\end{equation}
where P is the highest priority that can exist in the system.
\newline
\newline
For the above set of tasks, we get 18 constraints from Condition 1, 6 constraints from Condition 2, and 3 constraints from 
Condition 3 and finally 3 constraints from Condition 4. To solve the above problem, we get 30 constraints in all. Table \ref{} 
shows the number of high priority tasks that can be executed for a bank of size m.
 


