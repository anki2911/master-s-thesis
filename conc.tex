\chapter{Conclusion}
In this dissertation, we study the problem of memory scheduling for mixed criticality systems. We first take up the problem of deciding the required number of banks that can ensure a given set of mixed criticality tasks can meet their deadlines. We study both the decision and optimization problems for the same. Following this, we take up the issue of bank scheduling. 
The major problem with the conventional open row policy at the DRAM is that most of the high criticality tasks fail to meet their 
deadlines while waiting for memory access if not scheduled and executed within their deadlines. We propose a heuristic to address this limitation. Experimental results 
on a number of task sets with varying characteristics show the efficiency of our methods. We believe our research will open up future avenues in memory scheduling of mixed criticality systems. 