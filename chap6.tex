\chapter{Context Rule Modeling Using Linear Temporal Logic} \label{chap6}
\noindent
In this chapter, we present the extension of our work with linear temporal logic.
It is not always possible to express a context rule in simple propositional logic.
In some cases, the server may need to take a decision by observing environmental parameters across time.
Let us consider the following example.
For activating the fire alarm, the server needs to have the following two conditions satisfied:
smog comes out and within the next two minutes the temperature rises beyond a threshold.
In this example, the server needs the information of temperature for more than one cycle.
Simple propositional logic cannot help in this case. So, we introduce a higher order logic,
called linear temporal logic (LTL). We already discussed the syntax and semantics for linear temporal logic 
in Chapter~\ref{chap2}. 
In subsequent sections, we formally discuss the task of context rule modeling using temporal logic and 
the optimization strategies based on the context rules.


\section{Context Rule Modeling Using Linear Temporal Logic}\label{seclc}
\noindent
In this section, we present the syntax of the context rule in linear temporal logic. 
To model temporal dependencies in our context rule, we look at time in discrete steps, as is usually done in the temporal world. In other
words, time advances in discrete steps of ones and this allows us to capture dependencies across multiple time cycles. 
We now define the notion of a context rule and its associated actions.
\begin{definition}{\textbf{Context Rule :}} 
 A context rule is a Boolean combination of data predicates over time, 
 expressed in linear temporal logic. 
 \hfill $\blacksquare$
\end{definition}
Here, we consider two types of context rules.
\begin{itemize}
\item \textbf{B}-type context rule : These rules are expressed using the bounded fragments 
(Next, Bounded Until, Bounded Eventually, Bounded Always) of linear temporal logic.
 \item \textbf{S}-type context rule : These are safety rules, typically expressed using the unbounded operators (Always, Eventually, Until) 
 of linear temporal logic. These rules are always expected to be {\em{true}}. 
 A violation of a safety rule necessitates an action on the server side.
 
\end{itemize}
\begin{definition}{\textbf{Action :}} 
 It is the resulting actuation taken by the server with the violation of S-type context rules and with the true evaluation 
 of the B-type context rules. 
 \hfill $\blacksquare$
\end{definition}
\vspace{1.5mm}
\begin{definition}{\textbf{Context-Triggered Action :}} 
 Context-Triggered Action is an expression of the form Context Rule : Action. 
 \hfill $\blacksquare$
\end{definition}
\vspace{1.5mm}
\begin{definition}{\textbf{Temporal Depth of a Context Rule :}} 
 The temporal depth of a B-type context rule is the maximum levels of nesting of the temporal operators in the rule.  
 \hfill $\blacksquare$
\end{definition}
\vspace{1.5mm}
\noindent
Note that, a context rule in simple propositional logic has temporal depth 0.
The temporal depth of a S-type context rule is infinite.

\begin{example}
 Following is an example of a context-triggered action with B-type context rule: \\
 \noindent
 Context rule (${\cal{C}}_1$): The water level in the river is greater than the danger level and raining since 5 hours\\
 Action (${\cal{A}}_1$): Send flood alert. \\
 Context-Triggered Action: ${\cal{C}}_1$ : ${\cal{A}}_1$ \\
 Data predicate $p_1$ : The water level in the river $>$ the danger level ($t_1$); \\
 Data predicate $p_2$ : Whether it is raining; \\ 
 ${{\cal{C}}_1} = p_1 . G^{(5)}(p_2)$;\\ 
 The action gets fired with the true evaluation of the context rule.
 The temporal depth of the context rule is 5.  
 
 \noindent
 Following is an example of a context-triggered action with S-type context rule: \\
 Context rule (${\cal{C}}_2$): The water level in the storage is greater than a minimum threshold level until it is raining\\
 Action (${\cal{A}}_2$): Send shortage of water alert. \\
 Context-Triggered Action: ${\cal{C}}_2$ : ${\cal{A}}_2$ \\
 Data predicate $p_3$ : The water level in the river $>$ the minimum threshold level ($t_2$); \\
 Data predicate $p_4$ : Whether it is raining; \\
 ${{\cal{C}}_1} = p_3 U p_4$;\\ 
 The action gets fired with the violation of the context rule.
 The temporal depth of the context rule is infinite. 
\end{example}


\section{Optimization Strategies}\label{seclo}
\noindent
In this section, we discuss optimization strategies for context rules in LTL.

\subsection{Optimizations based on effective utility:}
\noindent
Our first analysis is based on the derivative of a context rule.
We have already defined the notion of derivative in Chapter~\ref{chap3}.
The definition, which we gave earlier, is valid for a Boolean formula. 
Now, our set of context rules are specified in linear temporal logic.
It is known that a bounded linear temporal logic formula can be 
unfolded as a Boolean logic formula using the notion of time frame expansion,
wherein multiple copies of the variable s are created to capture their existence over time. 
This mechanism works perfectly for our B-type context rules and hence they can be unfolded as Boolean formulae over time.
However, for S-type context rules, which are typically unbounded we can adopt a temporal upper bound and expand the formulae
similarly. This is how we handle the context rules in our framework at the cost of a blow up in the number of variables. 
Once we have the Boolean representation, we can now use the optimization operations defined in Chapter~\ref{chap3} 
in a straight forward way.

\begin{example}
Let us consider a few examples of unfolding for B-type context rules.\\
${\cal{C}}_1 = F_{[0,3]}(p) = p^{(0)} + p^{(1)} + p^{(2)} + p^{(3)}$;\\
${\cal{C}}_2 = G_{[0,3]}(q) = q^{(0)} . q^{(1)} . q^{(2)} . q^{(3)}$;\\
${\cal{C}}_3 = p U_{[0,3]} q = q^{(0)} + p^{(0)}.q^{(1)} + p^{(0)}.p^{(1)}.q^{(2)} + p^{(0)}.p^{(1)}.p^{(2)}.q^{(3)}$
\hfill$\blacksquare$
\end{example}

\noindent
In this work, we assume a 
user specified bound $t$ on the number of steps for 
which the unfolding needs to be performed. 
In the general setting, to have a conclusive decision 
about the derivative of a given LTL formula ${\cal C}$, we need to 
look up to bound $t$, where $t$ is $4 ^{1 + length ({\cal C})}$~\cite{r16}. 
The choice of the bound as an exponent of the length of the formula 
follows from the Ultimately Periodic Model Theorem~\cite{r16} for LTL.
However, in most practical 
cases, the user may have some idea on the upper bound for which 
he wishes to perform the derivative analysis.

\noindent
However, if we follow this approach, there is a blowup cost we have to pay.
The size of the formula can be the order of $d*k$ in the worst case, 
where $k$ is the number of predicates and $d$ is the temporal depth of the context rule. 
Intuitively, in the worst case, by unfolding a LTL formula we will have $d$ copies of each of the k predicates in its unfolded form.
Hence, the 
maximum number of computations required for derivative analysis of a context rule is also $d*k$  in the worst case.
Moreover, in each computation we have to deal with the large size of the formula.
But this approach sometimes leads to computational overhead.
%Consider $\cal{C}$ be a context rule. We want to perform the derivative of $\cal{C}$ with respect to $p^{(i)}$, i.e., a 
%predicate $p$ at $i^{th}$ time instant.

\begin{comment}
\begin{example}
 Let, ${\cal{C}} = q + G^{(5)}(p)$ be a context rule. We want to find the derivative with respect to $p^{(3)}$.
 ${\cal{C}} = q + p^{(3)}. (p^{(0)}.p^{(1)}.p^{(2)}.p^{(4)}.p^{(5)})$.
 Then we get the following after recursion:\\
 $\partial{\cal{C}}/\partial{p^{(3)}} = {\bar{q}}.p^{(0)}.p^{(1)}.p^{(2)}.p^{(4)}.p^{(5)}.\partial p^{(3)}/\partial{p^{(3)}}$
 \hfill$\blacksquare$
\end{example}
\vspace{1.5mm}
\end{comment}




%\begin{figure}[!htb]
%\begin{center}
%\includegraphics[height=1.2in,width=3.2in]{DA.png}
%\label{da}
%\caption{Critical \& Independent Predicates Analysis through Derivative}
%\end{center}
%\end{figure}

%Finally we discuss how the derivative analysis of a context rule helps us in message transmission reductions. 
%Obviously, the predicates which are not required to evaluate any of the context rule defined in the server,
%sensor can stop observing those predicates. On the other hand, the predicates which are critical for the 
%evaluation of a context rule always need to monitor and every change of those predicates need to transmit to
%the server.

%\begin{example}
% /*EXAMPLE*/
%\end{example}


\subsection{Optimizations based on co-factors:}
\noindent
In this section, we discuss the optimization techniques based on the co-factor analysis of a context rule. 
We expand the context rules as we discussed in the previous section and then do the co-factor analysis.
The optimization strategy is same as we explained in Chapter~\ref{chap3}.
The only difference is that here we have to do co-factor analysis across time cycles.
In this regard we discuss two models of co-factor analysis.

\noindent
{\bf{Without Memory Model : }}

\noindent
In this model, we assume that the sensor has no memory to keep the records for a data predicate.
In this model, if the current snapshot matches with an interest set of a context rule, the sensor needs to 
transmit the changes of the values of the predicates belonging to that interest set. Along with those predicates the
sensors may as well need to transmit some more predicates as explained in the example below.

\begin{example}
 Let, ${\cal{C}} = p_1 + Xp_2 + XXp_3$ be a context rule. Let us assume at a particular time instant, $p_2 = 1$. Now,
 $\{Xp_2\}$ is an interest set for $\cal{C}$. So, as long as the value of $p_2$ is true the server can take an action. 
 In this situation, if $p_2$ becomes false, to take an action, the server may
 require the value of $p_1$ in the previous cycle. Since the sensors have no memory, it is require to 
 transmit the value of $p_1$ along with $p_2$ in each cycle, though $p_2$ is enough to take the evaluation decision for ${\cal{C}}$.
 \hfill$\blacksquare$
\end{example}


\noindent
{\bf{With Memory Model : }}

\noindent
In this model, we assume that the sensor has sufficient memory to keep the records for a data predicate.
In this model, if the current snapshot matches with an interest set of a context rule, the sensor needs to 
transmit the changes of the values of the predicates belonging to that interest set only. In this case it may be noted,
we can avoid extra transmissions needed in the previous mode.

\noindent
We consider a context rule $\cal{C}$ with temporal depth $d$ and $k$ predicates.
If the minimum nesting depth of a predicate $p$ be $d_p$, we need
$(d - d_1)$ memory spaces at the sensor end to store the value of $p$ for the evaluation of $\cal{C}$.
This computation has to be performed for each predicate with respect to each context rule. Finally, the maximum 
memory space for each predicate across multiple context rules will have to be provision at the sensor end.

\begin{example}
Let, ${\cal{C}}_1 = p_1 + Xp_2$, ${\cal{C}}_1 = p_2 + XXp_3$. For ${\cal{C}}_1$ we need one memory space for $p_1$,
while for ${\cal{C}}_2$ two memory spaces are required for $p_1$. Therefore, for $p_1$, the sensor has to provision two memory spaces.
 \hfill$\blacksquare$
\end{example}

\begin{comment}
\noindent
Now we formally propose an algorithm to find the memory requirement of each predicate defined in the server.
%====================================
\begin{algorithm}
  \caption{MemoryRequirementAnalysis}
  \begin{algorithmic}[1]
  \State {Input:  ${\cal{P}}=\{p_1,p_2,\ldots,p_m\}, {\hat{\cal{C}}}=\{{\cal{C}}_1,{\cal{C}}_2,\ldots,{\cal{C}}_k\}$}
  \State {Output: ${\cal{M}}_1, {\cal{M}}_2,\ldots, {\cal{M}}_m$}\\
  \Comment{\emph{Memory space is required by each predicate}}
  \State Initialize ${\cal{M}}_1, {\cal{M}}_2,\ldots, {\cal{M}}_m$ by 0.
  \For{$c \in {\hat{\cal{C}}}$}
    \For{$p_i \in {\cal{P}}$}
      \State Calculate time instant $(d_1)$, at which $p_i$ is required to compute $c$ for the first time.    
      \If{Temporal Depth of $c -d_1 >   {\cal{M}}_i$} 
	\State ${\cal{M}}_i =$ Temporal Depth of $c -d_1$ 
      \EndIf
    \EndFor
  \EndFor
 \end{algorithmic}
 \label{algo1}
\end{algorithm}
%====================================
\end{comment}

\noindent
\paragraph*{The overall scheme: }

\noindent
Here, we discuss the overall approach for analyzing a context rule. The context rule, expressed in LTL is unfolded
to its corresponding Boolean formula first.
In the derivative analysis phase, the server identifies the set of predicates which are not required to evaluate any of the context rules.
Accordingly, the sensors stop monitoring these predicates.
The server does the co-factor analysis as discussed in Chapter~\ref{chap3} in the next phase. 
According to with/ without memory model,
the server takes the decision at runtime which predicates are required for current context computation.
The remaining optimization strategies are similar to what we discussed in Chapter~\ref{chap3}.




