\subsection{The Overall Scheme} \label{sec5}
\noindent
In this section, we present a top-down view of our proposed strategies. 
In the initial instant, the sensors send all the data predicates, whose derivatives are not 0, to the server.
Next time onwards, the sensors send the data predicates to the server only if the server asked for the data predicates.
When the server notifies the sensor for a particular data predicate, the sensor sends the predicate immediately and 
next time it sends the predicate to the server only if there is a 
change in the value of the predicate, i.e., in 
asynchronous mode. 

\noindent
The context rule evaluation semantics at the server end is as follows: \\
The server computes the predicates required to evaluate all 
the context rules in the next instance based on the current snapshot and accordingly notifies it to the sensor. 
As in the synchronous mode, the transmission follows the query-response model but unlike synchronous mode, instead of sending 
all the predicates in each round, the sensors send the predicates only if they are notified to send. 
Again, unlike asynchronous mode, the sensor does not send the predicates all the time, but only when the values of the 
predicates change. Hence, the transmission mode basically falls between the synchronous and asynchronous transmission 
mechanisms.
The step by step transmission optimization mechanism for our strategy is shown in Figure~\ref{flow_chart}.
We are given a set $\hat{{\cal{C}}}$ of context rules over a set of Boolean predicates ${\cal{P}}$. 
The server performs the following pre-processing steps:

\begin{figure}%[htb]
\begin{center}
\fbox{
\includegraphics[height=5in,width=5in]{fig0.png}

}
\caption{Transmission Optimization Mechanism}
\label{flow_chart}
\end{center}
%\label{fc}
\end{figure}


\vspace{2 mm}

\noindent
 {\bf{Step 1:}} Use Derivative analysis to compute the set 
 ${\cal{P}}_1$, the set of predicates that is critical to the server, and 
 ${\cal{P}}_2$, the set of predicates not required for any context rule by the server.
 Based on the result, the server informs the sensors to stop monitoring the data variables 
 corresponding to the data predicates belonging to ${\cal{P}}_2$ and also to transmit all 
 changes of the data predicates belonging to ${\cal{P}}_1$. 
 
 \noindent
The formal procedure for derivative analysis is shown in Algorithm~\ref{algo1}.

%====================================

\begin{algorithm}[!htb]
  \caption{DerivativeAnalysis}

  \begin{algorithmic}[1]
    %\Procedure{DerivativeAnalysis} {${\cal{P}}, {\cal{C}}$}\\
\State { Input:  ${\cal{P}}=\{p_1,p_2,\ldots,p_m\}, {\cal{C}}=\{{\cal{C}}_1,{\cal{C}}_2,\ldots,{\cal{C}}_k\}$}
\State { Output: ${\cal{P}}_1, {\cal{P}}_2, {\cal{P}}_3$}
\For{$p \in {\cal{P}}$}
  \For{$c \in {\cal{C}}$}
    \State Calculate $\partial{c}/\partial{p}$
    \If{$\partial{c}/\partial{p} = 1$} %for some $c \in {\cal{C}}$ and $u \in {\cal{V}}$}
      \State ${\cal{P}}_1 = {\cal{P}}_1 \cup \{p\}$ 
          
    \EndIf
  \EndFor
  \If{$\partial{c}/\partial{p} = 0,\text{ } \forall c \in {\cal{C}}$}  
      \Comment{\emph{Useless for all rules}}
      \State ${\cal{P}}_2 = {\cal{P}}_2 \cup \{p\}$
  \EndIf
\EndFor
\State ${\cal{P}}_3 = {\cal{P}} \setminus ({\cal{P}}_1 \cup {\cal{P}}_2)$      
%\EndProcedure 
  \end{algorithmic}
  \label{algo1}
\end{algorithm}


%====================================


\begin{comment}

 \begin{boxedalgorithmic}%[1]    
 \Procedure{DerivativeAnalysis} {${\cal{P}}, {\cal{C}}$}\\
{ Input:  (${\cal{P}}=\{p_1,p_2,\ldots,p_m\}, {\cal{C}}=\{{\cal{C}}_1,{\cal{C}}_2,\ldots,{\cal{C}}_k\}$)}\\
{ Output: (${\cal{P}}_1, {\cal{P}}_2, {\cal{P}}_3$)}
\For{$p \in {\cal{P}}$}
  \For{$c \in {\cal{C}}$}
    \State Calculate $\partial{c}/\partial{p}$
    \If{$\partial{c}/\partial{p} = 1$} %for some $c \in {\cal{C}}$ and $u \in {\cal{V}}$}
      \State ${\cal{P}}_1 = {\cal{P}}_1 \cup \{p\}$ 
          
    \EndIf
  \EndFor
  \If{$\partial{c}/\partial{p} = 0,\text{ } \forall c \in {\cal{C}}$}  \\
      \Comment{\emph{Useless for all rules}}
      \State ${\cal{P}}_2 = {\cal{P}}_2 \cup \{p\}$
  \EndIf
\EndFor
\State ${\cal{P}}_3 = {\cal{P}} \setminus ({\cal{P}}_1 \cup {\cal{P}}_2)$      
\EndProcedure

\end{boxedalgorithmic}

\label{algo1}
\end{comment}

%\\
\vspace{2 mm}
\noindent
 {\bf{Step 2:}} Evaluate the positive and negative co-factors of all 
 	 context rules with respect to 
 	 each predicate belonging to (${\cal{P}} \setminus {\cal{P}}_2$) in the cases where 
 	 $\partial{c}/\partial{p} \ne 1$.
 	 
 \vspace{2 mm}
 \noindent
 {\bf{Step 3:}} Analyze the two variable co-factors for each context rule 
 	 ${{\cal{C}}}$ using the  
     set of predicates such that ${\cal{C}}_p$ and ${\cal{C}}_{\bar{p}}$ 
 	 are not constant for all $p$ in that predicate set.
  \noindent
In Algorithm~\ref{algo2} we explain the formal procedure for co-factor analysis.
 
 %++++++++++++++++++++++++++++++++++++++
 
 
\begin{algorithm}[!htb]
  \caption{Co-factorAnalysis}

  \begin{algorithmic}[1]
\State{ Input:  ${\cal{P}}_1, {\cal{P}}_3, {\cal{C}}=\{{\cal{C}}_1,{\cal{C}}_2,\ldots,{\cal{C}}_k\}$}\\
\Comment{\emph{Single Variable Co-factor Analysis}}
 \For {$c \in {\cal{C}}$}
  \For {$p \in {\cal{P}}_1 \cup {\cal{P}}_3$ and $\partial{c}/\partial{p} \neq 1$}
    \State calculate $c_p$ and $c_{\bar{p}}$
    \State Find the set ${\cal{P}}_{c_{p}}$, ${\cal{P}}_{c_{\bar{p}}}$, the set of all variables 
    required to evaluate $c_p$ and $c_{\bar{p}}$ 
  \EndFor
 \EndFor\\
\Comment{\emph{Multiple Variable Co-factor Analysis}}
  \For {$c \in {\cal{C}}$}
   \For {$p, q \in ({\cal{P}}_1 \cup {\cal{P}}_3)$ and $p\neq q$ and $\partial{c}/\partial{p} \neq 1$ 
   and $\partial{c}/\partial{q} \neq 1$}
	\State calculate $c_{pq}$, $c_{\bar{p}q}$, $c_{p\bar{q}}$ and $c_{\bar{p}\bar{q}}$\\
  \indent \Comment{\emph{If the co-factor of $c$ with respect to any of the literals in the
   set $\{p,\bar{p},q,\bar{q}\}$ }} is ~\\
   \indent~~~~~~~~~ {\emph{constant, we do not calculate the double variable co-factor involving that literal}}
    %\indent \Comment{\emph{that literal}} 
  %\indent \indent \Comment{\emph{   ~to any of the literals in the set}} \\
  %\indent \indent \Comment{\emph{ $\{p,\bar{p},q,\bar{q}\}$}} is constant, then we\\
  %\indent \indent \Comment{\emph{~ ~~~~do not calculate the double}}\\
  %\indent \indent \Comment{\emph{variable co-factor involving that}}\\
  %\indent \indent \Comment{\emph{ literal}}
	\State Find the sets ${\cal{P}}_{c_{pq}}$, ${\cal{P}}_{c_{\bar{p}q}}$, ${\cal{P}}_{c_{p\bar{q}}}$  
	and ${\cal{P}}_{c_{\bar{p}\bar{q}}}$, the set of predicates required to evaluate 
       \indent~~ $c_{pq}$, $c_{\bar{p}q}$, $c_{p\bar{q}}$ and $c_{\bar{p}\bar{q}}$      
      \EndFor  
   \EndFor
 %\EndProcedure
%\label{algo2}
  \end{algorithmic}
  \label{algo2}
\end{algorithm} 
 %++++++++++++++++++++++++++++++++++++++

 \begin{comment}
\begin{boxedalgorithmic}[1] 
\Procedure{Co-factorAnalysis}{${\cal{P}}_1, {\cal{P}}_3, {\cal{C}}$}\\
{ Input:  (${\cal{P}}_1, {\cal{P}}_3, {\cal{C}}=\{{\cal{C}}_1,{\cal{C}}_2,\ldots,{\cal{C}}_k\}$)}\\
%{ Output: ${\cal{V}}_o = \{{\cal{V}}_{c_1},{\cal{V}}_{c_2},\ldots,{\cal{V}}_{c_k}\}$}
\Comment{\emph{Single Variable Co-factor Analysis}}
 \For {$c \in {\cal{C}}$}
  \For {$p \in {\cal{P}}_1 \cup {\cal{P}}_3$ and $\partial{c}/\partial{p} \neq 1$}
 %so that the derivative analysis of $u$ with respect to $c$ is not 1 }
    \State calculate $c_p$ and $c_{\bar{p}}$
    \State Find the set ${\cal{P}}_{c_{p}}$, ${\cal{P}}_{c_{\bar{p}}}$, the set of all 
    \indent \indent ~variables required to evaluate $c_p$ and $c_{\bar{p}}$ 
  \EndFor
 \EndFor\\
\Comment{\emph{Multiple Variable Co-factor Analysis}}
  \For {$c \in {\cal{C}}$}
   \For {$p, q \in ({\cal{P}}_1 \cup {\cal{P}}_3)$ and $p\neq q$ and $\partial{c}/\partial{p} \neq 1$ 
   \indent \indent ~~and $\partial{c}/\partial{q} \neq 1$}
      %the derivative analysis of $u, v$ with respect to $c$ are not 1}
	\State calculate $c_{pq}$, $c_{\bar{p}q}$, $c_{p\bar{q}}$ and $c_{\bar{p}\bar{q}}$\\
  \indent \Comment{\emph{If the co-factor of $c$ with respect}} \\
  \indent \indent \Comment{\emph{   ~to any of the literals in the set}} \\
  \indent \indent \Comment{\emph{ $\{p,\bar{p},q,\bar{q}\}$}} is constant, then we\\
  \indent \indent \Comment{\emph{~ ~~~~do not calculate the double}}\\
  \indent \indent \Comment{\emph{variable co-factor involving that}}\\
  \indent \indent \Comment{\emph{ literal}}
	\State Find the set ${\cal{P}}_{c_{pq}}$, ${\cal{P}}_{c_{\bar{p}q}}$, ${\cal{P}}_{c_{p\bar{q}}}$ and 
	${\cal{P}}_{c_{\bar{p}\bar{q}}}$,
      \indent \indent~ the set of all variables required to evaluate 
      \indent \indent~ $c_{pq}$, $c_{\bar{p}q}$, $c_{p\bar{q}}$ and $c_{\bar{p}\bar{q}}$  
      \EndFor  
   \EndFor
 \EndProcedure
\end{boxedalgorithmic}   
\label{algo2}
\end{comment}
%\\
%\\\\
\vspace{2 mm}
\noindent
 {\bf{Step 4:}} Sensor Data Aggregation: Consider the set 
 ${\cal{P}}_3 = ({\cal{P}} \setminus ({\cal{P}}_1 \cup {\cal{P}}_2))$. The server can 
 instruct the sensor which predicates can be transmitted by ORing or ANDing. 
 Let the new set be ${\cal{P}}_4$, where some predicates belonging to ${\cal{P}}_3$ are 
 combined into one. Though they are different data parameters while observing, 
 they are combined while transmitting.
 
 \vspace{2 mm} 
 \noindent
 {\bf{Step 5:}} For each context rule ${{\cal{C}}}$, 
 		find two sets of predicates in which the context rule is positive 
        and negative unate such that the derivative of the predicate that belongs to these two sets is not constant.
        %such that $\forall v$ belong to that 
        %sets $\partial{c}/\partial{v}$ is not equal to constant.
 
%\\
 %\\       %\end{itemize}
 
 \noindent
We describe the formal procedure for computing the positive and negative unate sets in Algorithm~\ref{algo3}.
 %**********************************
 
 
\begin{algorithm}[!htb]
  \caption{UnateAnalysis}

  \begin{algorithmic}[1]
  \State { Input:  ${\cal{P}}_1, {\cal{P}}_3, {\cal{C}}=\{{\cal{C}}_1,{\cal{C}}_2,\ldots,{\cal{C}}_k\}$}
%{ Output: $\cal{R}$}
%\begin{enumerate}
 \For{$c \in {\cal{C}}$}
  \For{$p \in {\cal{P}}_1 \cup {\cal{P}}_3$ such that $\partial{c}/\partial{p} \neq 1$} 
    \If{$c$ is positive unate in $p$}
      \State put $p$ in $Pos_c$ \Comment{\emph{Set of positive unate}}
    \ElsIf{$c$ is negative unate in $p$} 
      \State put $p$ in $Neg_c$ \Comment{\emph{Set of negative unate}}
    \EndIf
  \EndFor
 \EndFor
  \end{algorithmic}
  \label{algo3}
\end{algorithm}


 
 %**********************************

 
 \begin{comment}
 
\begin{boxedalgorithmic}[1] 
\Procedure{UnateAnalysis}{${\cal{P}}_1, {\cal{P}}_3, {\cal{C}}$}\\
{ Input:  (${\cal{P}}_1, {\cal{P}}_3, {\cal{C}}=\{{\cal{C}}_1,{\cal{C}}_2,\ldots,{\cal{C}}_k\}$)}
%{ Output: $\cal{R}$}
%\begin{enumerate}
 \For{$c \in {\cal{C}}$}
  \For{$p \in {\cal{P}}_1 \cup {\cal{P}}_3$ such that $\partial{c}/\partial{p} \neq 1$} 
    \If{$c$ is positive unate in $p$}
      \State put $p$ in $Pos_c$ 
    \ElsIf{$c$ is negative unate in $p$} 
      \State put $p$ in $Neg_c$
    \EndIf
  \EndFor
 \EndFor
 %\item At execution time $\forall{c} \in {\cal{C}}$, if $c=1$
 %find the set ${\cal{V}}_c = {\cal{V}}_c \setminus \{u\}$, where $u=0$ and $u \in Pos_c$;
 %if $c=0$
 %find the set ${\cal{V}}_c = {\cal{V}}_c \setminus \{u\}$, where $u=1$ and $u \in Neg_c$.  
 % $R = {\cal{V}}_3 \setminus ({\cup_{c \in {\cal{C}}}}{\cal{V}}_c)$, the set which is not currently required by the server.
%\end{enumerate}
\EndProcedure
\end{boxedalgorithmic}   
\label{algo4}
\\
\\\end{comment}
\vspace{2 mm}
\noindent
The following is an execution time computation:

\vspace{2 mm}
%\begin{itemize}
\noindent
{\bf{Step 6:}} Find the set of predicates (${\cal{P}}_c$) to evaluate each context rule (${{\cal{C}}}$) 
 		depending on the current snapshot. Then the server
	    informs the appropriate sensors to stop transmitting predicates 
	    belonging to ${\cal{P}}_3 \setminus (\cup _{c \in {\cal{C}}}{{\cal{P}}_c})$.
%\end{itemize}




