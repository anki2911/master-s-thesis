The use of temporal logics~\cite{r17} in verification was proposed by Pnueli in a seminal
paper~\cite{r14}. Since then several different logics have been proposed
for specifying temporal properties. 
All these logics use the two basic temporal operators --
{\em next} and {\em until}. Some of these logics also use additional 
temporal operators that can be derived out of the basic two. The logics differ
in terms of how we are allowed to mix these operators to express the desired
formula.

\noindent
In this section, we introduce the popular
temporal operators and the logics that are built around them. In this part
we also introduce some formalisms in an intuitive way that show us how
these logics are interpreted over time.

\subsubsection{The basic temporal operators} \label{sec2.1} 
\index{temporal operators}

\noindent
The formal introduction to a language has two main parts, namely the 
{\em syntax} and the {\em semantics}. The syntax defines the {\em grammar}
of the language -- it tells us how we may construct properties using the
basic set of signals and operators. The semantics define the {\em meaning}
of the properties.

\noindent
The semantics of the traditional temporal logics were defined over {\em closed
systems}, which are finite state machines without any inputs. This tradition
has been followed in languages such as SVA and PSL as well -- there is no
distinction between input and non-input variables in these languages. At this
point we present the semantics of these languages in the traditional form
over a non-deterministic finite state machine. Open systems (modules having
input bits) can be modeled by treating the input bits also as state bits. This
typically yield a non-deterministic state machine, since the choice of
inputs in the next state lies with the environment, and is not a function of
the present state.

\noindent
Suppose $J$ is a finite state machine having $k$ state bits. Each of the $2^k$
valuations of these state bits represent a {\em state} of the machine.
Let $S$ denote the set of these states. Let $R$ denote the {state transition
relation} of $J$. $R$ consists of pairs of states, $(s_i, s_j)$, where it
is possible to transit from state $s_i$ to state $s_j$. Finally, $J$ has a
start state $s$. Formaly we say that $J$ is a tuple 
$\langle S, s, R \rangle$.

\begin{figure}[htb]
\centering
%\includegraphics[height=2in,width=3.5in]{../temporal-logics/diag23.png}
\center
\caption{A sample finite state machine} \label{fig2.3}
\end{figure}


\begin{example} \label{ex2.1}
Fig~\ref{fig2.3} shows a 3-bit finite state machine. Let the state bits be
$n_0, n_1, n_2$. The state bits are shown on the nodes. The start state is
$s_0$. Fig~\ref{fig2.3} shows 5 states -- the remaining three states are
not reachable from the start state and are not shown. The circuit has
three outputs, which are functions of the state bits. These are:
\begin{eqnarray*}
p & = & n_0\ \lor\ n_1 \\
q & = & n_2 \\
r & = & \neg n_0\ \land\ \neg n_1\ \land\ \neg n_2
\end{eqnarray*}
The nodes of Fig~\ref{fig2.3} are labeled by the outputs that are true at
that state. We shall use this toy example to demonstrate the meaning of 
various temporal properties. $\Box$
\end{example}

\subsubsection{Intuitive explanation}
\noindent
To convey the semantics of the basic temporal operators, we first
introduce the notion of a {\em run} (alternatively, a {\em path} or a 
{\em trace}). A run, $\pi$, of $J$ is a sequence of states,
$\nu_0, \nu_1, \ldots$, where $s = \nu_0$ is the start of the run, and for
each $i$, $\nu_i$ represents a state in $S$, and $R$ contains a transition 
from the state represented by $\nu_i$ to the state represented by $\nu_{i+1}$.
In other words, the run is a sequence of states repesenting a valid sequence
of state transitions of $J$. For example the run,
$\pi = s_0, s_1, s_3, s_1, s_4, \ldots$, is one run of the state machine shown
in Fig~\ref{fig2.3}. States of the machine may be revisited in the run --
for example we have $\nu_1 = \nu_3 = s_1$ in $\pi$. The run, 
$\pi' = s_0, s_2, s_0, \ldots$, does not belong to this state machine, since
it has no transition from $s_2$ to $s_0$.

\noindent
Let us now consider the two fundamental temporal operators, namely {\em next}
and {\em until}, and a run $\pi = s_0, s_2, s_3, \ldots$.
\begin{description}

\item[Next operator: ] \index{temporal operators!next time} A property, 
	{\em next $f$}, is true at a state
	of a run iff the property $f$ is true at the next state on the run.
	For example, {\em next $q$} is false at the state, $s_0$, of the run,
	$\pi = s_0, s_2, s_3, \ldots$, since $q$ is false at the next state
	$s_2$. The property {\em next next $q$} is true at $s_0$,
	of $\pi$, because $q$ is true at $s_3$.

\item[Until operator: ] \index{temporal operators!until}
	A property, {\em $f$ until $g$}, is true at a
	state of a run iff the property $g$ holds on some future state, $z$,
	of the run, and the property $f$ holds on all states preceding $z$
	on the run. For example, the property, {\em $p$ until $q$}, is true
	at the start state of $\pi$, since $q$ is true at the state $s_3$
	and $p$ is true at the states $s_0, s_2$ preceding $s_3$ in $\pi$.
	The property, {\em $p$ until $r$}, is false on all paths of 
	Fig~\ref{fig2.3}, because no $r$-labelled state can be reached along
	a $p$-labelled path starting from $s_0$.

\end{description}
We now define the two other operators  
namely, {\em always} and {\em eventually}. To do this, we need the definitions
of the propositions, {\em TRUE} and {\em FALSE}. We say that the proposition,
{\em TRUE}, holds in all states, and the proposition, {\em FALSE}, is false
in all states.
\begin{description}

\item[Eventually operator: ] \index{temporal operators!eventually}
	A property, {\em eventually $f$}, is true at a state
	of a run iff the property $f$ holds on some future
	state in the run. Since the proposition, TRUE, holds on all states,
	we can express the {\em eventually} operator using the {\em until}
	operator as:
	\[ \mbox{\em eventually } f\ =\ \mbox{\em TRUE until } f \]
	The property, {\em eventually $q$}, holds on all runs starting from
	$s_0$ in Fig~\ref{fig2.3}. The property, {\em eventually $r$}, does
	not hold in the run which loops forever in the loop
	$s_0, s_2, s_3, s_0$.

\item[Always operator: ] \index{temporal operators!always}
	A property, {\em always $f$}, is true on a run iff
	the property $f$ holds on all states of the run. This is the same
	as saying that $\neg f$ never holds on the run. In other words we
	may write:
	\begin{eqnarray*}
	\mbox{\em always } f & = & \neg \mbox{\em eventually } \neg f \\
	\mbox{\em eventually } f & = & \neg \mbox{\em always } \neg f
	\end{eqnarray*}
	The first equation allows us to express the {\em always} operator
	using the {\em eventually} operator, and in turn, in terms of the
	{\em until} operator. The second, says: {\em sometimes is not never}
	-- there is a seminal paper with this title by Leslie 
	Lamport~\cite{r15}.

	The property, {\em always $p$} is true in the run which loops forever
	in the loop, $s_0, s_2, s_3, s_0$, in Fig~\ref{fig2.3}. The property
	is false in all other runs of the same state machine.

\end{description}
The duality between the {\em always} and {\em eventually} operators is not
surprising. \index{temporal operators!duality between} 
In fact, it is a variant of DeMorgan's Laws when we interpret
the properties over time. This is because:
\begin{eqnarray*}
\mbox{\em eventually } f & = & f\ \lor\ \mbox{\em next } f\ \lor\ 
				\mbox{\em next next } f\ \lor\ 
				\mbox{\em next next next } f\ \ldots\\
& = & \neg \left( \neg f\ \land\ \mbox{\em next } \neg f\ \land\ 
		\mbox{\em next next } \neg f\ \land\
		\mbox{\em next next next } \neg f\ \ldots \right) \\
& = & \neg \left( \mbox{\em always } \neg f \right)
\end{eqnarray*}

\noindent
{\em Linear Temporal Logic} (LTL) is the most popular among linear time
logics. We can define the syntax of linear temporal logic recursively as follows:
\begin{itemize}

\item All Boolean formulas over the state variables are LTL properties.

\item If $f$ and $g$ are LTL properties, then so are: $\neg f$, $Xf$, and
	$f\ U\ g$.

\end{itemize}
\noindent
We can also use the short-forms, $Fg$ for $true\ U\ g$, and $Gf$ for
$\neg (true\ U\ \neg f)$.
\subsubsection{Formal semantics} \index{temporal operators!semantics of}
\noindent
It is very important to know the formal semantics of a formal property
specification language. If the semantics is specified ambiguously, there may
be a gap between the property that the designer intends to express and the
formal property tool's interpretation of the property that she writes.
Bugs may hide in this gap thereby defeating the whole purpose of {\em formal}
property verification. Language lawyer volunteers who make up the working
groups of the language standards committees spend years debating
over the exact formal semantics of the languages that they standardize.
The goal of standardization is to ensure that languages with precise 
definitions are made available to improve communcation within the
industry.

\noindent
The problem with understanding formal semantics is that they are replete with
terse notations. It is widely suspected that the intimidating nature of the
notations used in existing literature on formal property verification is one
of the main deterants to its wider adoption in practice.

%\noindent
%In reality, the formal semantics of the temporal operators do not convey 
%anything more than what we have discussed already. Nevertheless it is
%worthwhile to present the formal semantics, not only for the sake of
%completeness, but also to familiarize ourselves with the notations that 
%appear in almost all texts on formal property verification, including some 
%chapters of this book.

\noindent
To start with, we use a set of short-forms. We use $X$ to denote
the {\em next} operator, $U$ to denote the {\em until} operator, $G$ to
denote the {\em always} operator, and $F$ to denote the {\em eventually} 
operator. $G$ means {\em globally with respect to time}, and $F$ means
{\em in the future}.

\noindent
Let $\pi = \nu_0, \nu_1, \ldots$ denote a run, and 
$\pi^k = \nu_k, \nu_{k+1}, \ldots$ denote the part of $\pi$ starting from
$\nu_k$. We use the notation $\pi \models f$ to denote that the property
$f$ holds on the run $\pi$. Given a run $\pi$, we also use the notation
$\nu_k \models f$ to denote $\pi^k \models f$. In other words, a property is
said to be true at an intermediate state of the run iff the fragment of the
run starting from that state satisfies the property. The formal semantics of
the basic temporal operators are as follows:
\begin{itemize}

\item $\pi$ $\models$ $Xf$ iff $\pi_1 \models f$
\item $\pi$ $\models$ $f\ U\ g$ iff $\exists j$ such that 
	$\pi_j\ \models\ g$ and $\forall i, 0 \leq i < j$ we have
	$\pi_i\ \models\ f$.

\end{itemize}
$Fg$ is a short-form for TRUE $U\ g$, and $Gf$ is a short-form for
$\neg F \neg f$.

\begin{comment}
\section{Logics for temporal specification} \label{sec2.2}
\noindent
Temporal logics tell us how we can create complex temporal properties by
putting together one or more temporal operators. There are broadly two
classes of these logics, namely {\em linear time logics} and {\em branching
time logics}. Linear time logics allow the specification of properties over
linear traces or runs of a finite state machine -- intuitively, we say that
the property holds on the machine if it holds on all runs of the machine.
Branching time logics allow the specification of properties over the 
computation tree created by a state traversal of the state machine.

\subsubsection{Linear Temporal Logic} \label{sec2.2.1}
\index{temporal logics!Linear Temporal Logic (LTL)}
\noindent
Designers and validation engineers typically express and interpret the
RTL in terms of the simulation semantics of the HDL. They are accustomed to
verifying the correctness of the RTL by checking certain behaviors over 
simulation traces. Therefore it is not suprising that linear time logics
have been the natural choice for design validation, and form the backbone
of most existing property specification languages, including Forspec, PSL
and SVA.



\noindent
The semantics of LTL is as follows. We say that the property $f$ holds
on a state machine, $J$, iff $f$ holds on all paths of the state machine
starting from its start state. The semantics of $f$ on a path is
as defined in the last section.

\noindent
Let us see some sample LTL properties obtained by using one or more temporal
operators. We refer to Fig~\ref{fig2.3}.
\begin{itemize}

\item The property $p\ U\ q$ is true in the state machine, since all paths
	from $s_0$ satisfy this property.

\item The property $Fq$ is true in the state machine, but the property
	$GFq$ is not true. This is because we have the path, 
	$\pi = s_0, s_1, s_4, \ldots$, which does not satisfy $Fq$ from
	$s_4$ onwards.

\item The property $p\ U\ (q\ U\ r)$ is not true in the state machine, 
	because it may get trapped in the loop, $s_0, s_2, s_3, s_0$.

\end{itemize}
\noindent
Fig~\ref{fig2.4} shows some sample LTL properties and some sample runs
that satisfy these properties.

\begin{figure}[htb]
\centering
\includegraphics[height=2.5in,width=4in]{../temporal-logics/diag24.png}
\center
\caption{Some sample LTL properties} \label{fig2.4}
\end{figure}

%\begin{comment}
\subsubsection{Computation Tree Logic} \label{sec2.2.2}
\index{temporal logics!Computation Tree Logic (CTL)}
\noindent
{\em Computation Tree Logic} (CTL) is a branching time temporal logic.
Properties described in this logic are interpreted over the {\em computation
tree} obtained by unfolding the state machine as a tree. We elaborate
on this shortly, but let us first study the basic features of this logic.

\begin{figure}[htb]
\centering
\includegraphics[height=2.5in,width=3.5in]{../temporal-logics/diag25.png}
\center
\caption{Some sample CTL properties} \label{fig2.5}
\end{figure}
\noindent
CTL has two {\em path quatifiers} in addition to the usual temporal operators.
These are the {\em existential} path quantifier $E$, and the {\em universal}
path quantifier $A$. 
\begin{itemize}

\item The property $A \varphi$ is true at a state, $\nu$, iff $\varphi$ is
	true on {\em all} runs starting from $\nu$.

\item The property $E \varphi$ is true at a state, $\nu$, iff $\varphi$ is
	true on {\em some} run starting from $\nu$.

\end{itemize}
\noindent
In CTL we have the restriction that every subformula of the form $Xf$, $Gf$,
$Fg$, and $f\ U\ g$ must be prefixed by an $E$ or $A$. Therefore, we may
define the language as:
\begin{itemize}

\item All Boolean formulas over the state variables are CTL properties.

\item If $f$ and $g$ are CTL properties, then so are: $\neg f$, $AXf$, 
	$EXf$, $A[f\ U\ g]$ and $E[f\ U\ g]$.

\end{itemize}
\noindent
We also have the usual short-forms $Fg$ for $true\ U\ g$, and $Gf$ for
$\neg (true\ U\ \neg f)$. Consequently, $EFg$, $AFg$, $EGf$, $AGf$ are CTL
properties. Fig~\ref{fig2.5} shows some sample CTL properties and some sample
computation trees that satisfy these properties.

\begin{figure}[htb]
\centering
\includegraphics[height=2.5in,width=4.3in]{../temporal-logics/diag26.png}
\center
\caption{The notion of a Computation Tree} \label{fig2.6}
\end{figure}
\noindent
What is the significance of a {\em computation tree}? Let us consider the
state machine of Fig~\ref{fig2.3}. We can unfold the state machine into 
the infinite tree shown in Fig~\ref{fig2.6}. Each path of this tree is a
run or a {\em computation} of the state machine. CTL allows us to define
properties over the nodes of this computation tree.

\noindent
For example, consider the property:
\[ A[p\ U\ E[q\ U\ r]] \]
This property is true at the start state $s_0$ of our state machine. This
follows from the fact that $E[q\ U\ r]$ is true at $s_1$ and $s_3$ (since
there is a $q$-labeled sequence of states to the $r$-labelled state $s_4$),
and every path from $s_0$ reaches one of these states through $p$-labelled
states. It is not possible to express this property in LTL.

\subsubsection{LTL versus CTL} \label{sec2.2.3}
\index{temporal logics!LTL versus CTL}
\noindent
{\em Can all LTL properties be expressed in CTL using the universal path
quantifier, $A$?}
The answer is, no. For example, the LTL property, $FGp$, is not 
equivalent to the CTL property, $AFAGp$. Fig~\ref{fig2.7} shows an example
which satisfies $FGp$ but does not satisfy $AFAGp$.

\begin{figure}[htb]
\centering
\includegraphics[height=2.5in,width=4in]{../temporal-logics/diag27.png}
\center
\caption{Module M satisfies FGp, but not AFAGp} \label{fig2.7}
\end{figure}
\noindent
{\em When do we use a linear time logic, and when do we use a branching time
logic?} This is a matter of considerable debate, and is hardly agreed upon.
However, experience shows that linear time logics are the natural choice
for black-box testing. For example, while specifying the behavior of a
module, we can write linear time properties over its interface signals 
without knowing the internal state machine of the module. 

\noindent
On the other hand, branching time logics are useful for verifying properties
over a given state machine. For example, when we develop an automotive control
system we may typically model the control system as an abstract state machine,
verify whether this control system satisfies certain safety properties and
then expand the abstract state machine into the actual control system.
When a branching time property fails, we must interpret the failure in terms
of the actual states of the system, which is not possible in black-box
testing.

\noindent
The temporal logic CTL$^*$ combines the expressive power of linear and
branching time temporal logics. CTL and LTL are fragments of CTL$^*$. There
has been several interesting extensions of these languages that demonstrate the
tradeoff between the expressive power of the language and the complexity of
model checking (that is, formally verifying) properties specified in these
languages.
\end{comment}

\subsubsection{Bounded temporal logics} \label{sec2.2.4}
\index{temporal logics!real time}
\noindent
The temporal operators discussed so far, namely {\em next} (X), {\em until}
(U), {\em always} (G), and {\em eventually} (F), are {\em temporal} because
they can define sequences of events over time. Significantly, none of these 
operators with the exception of the {\em next} operator, attempt to 
{\em quantify} time. For example the property, {\em eventually $f$}, requires
$f$ to be true in future, but does not specify any time bound by which $f$
needs to be true.

\noindent
Real time temporal operators \index{temporal operators!real time}
are intuitively simple extensions of the basic
{\em untimed} temporal operators where we annotate the operator with a time
bound. The real time extensions of CTL and LTL simply use these bounded 
operators (as well as the unbounded ones). 
\begin{description}

\item[The bounded Until operator: ] \index{temporal operators!bounded until}
	The property $f U_{[a,b]} g$ is true on
	a run, $\pi = s_0, s_1, \ldots$, iff there exists a $k$, 
	$a \leq k \leq b$ such that $g$ is true at $s_k$ on $\pi$, and $f$
	is true on all preceding states, $s_0, \ldots, s_{k-1}$. Formally,
	\[ \pi\ \models\ f\ U_{[a,b]}\ g \mbox{ iff } \exists k,
	a \leq k \leq b, \nu_k\ \models\ g \mbox{ and } 
	\forall i, 0 \leq i < k \mbox{ we have } \nu_i\ \models\ f \]
	The bounded LTL property $p\ U_{[1,3]}\ q$ is true at the state
	$s_0$ of Fig~\ref{fig2.3}. The bounded CTL property:
	\[ A[ p\ U_{[1,3]}\ E[ q\ U_{[1,2]}\ r]] \]
	is also true at $s_0$. This is because $s_3$ and $s_1$ satisfy
	$E[ q\ U_{[1,2]}\ r]$ (since they can reach $s_4$ within the time
	bound $[1,2]$), and we reach $s_1$ or $s_3$ along all paths from
	$s_0$ within the time bound $[1,3]$.

\item[The bounded Eventually operator: ] The property $F_{[a,b]} g$ is true
	on a run, $\pi = s_0, s_1, \ldots$, iff there exists a $k$,
	$a \leq k \leq b$ such that $g$ is true at $s_k$ on $\pi$.
	For example, the bounded LTL property $F_{[1,3]} q$ is true at the
	state $s_0$ of Fig~\ref{fig2.3}. The bounded CTL property 
	$EF_{[2,4]} r$ is true at $s_0$. The property $F_{[a,b]} g$ is
	equivalent to the bounded-until property, $true\ U_{[a,b]}\ g$. 

\item[The bounded Always operator: ] The property $G_{[a,b]} f$ is true
	on a run, $\pi = s_0, s_1, \ldots$, iff $f$ is true in every
	state in the sequence, $s_a, \ldots, s_b$. The bounded CTL property
	$EG_{[3,9]} q$ is true in the state $s_0$ of Fig~\ref{fig2.3} --
	consider the run from $s_0$ through $s_2$ which alternates between
	$s_3$ and $s_1$ at least 8 times. The bounded LTL property
	$G_{[0,1]} \neg r$ is true at $s_0$ since no run can reach $s_4$
	is less than 2 cycles.

\end{description}
\noindent
Real time operators are extremely useful in practice. Most design properties
have a well defined time bound, and must be satisfied within that time.

\noindent
Since the real time operators deal with finite bounds, $a$ and $b$, they
can be expressed in terms of the $X$ operator. For example, the property
$F_{[2,4]} q$ can be rewritten as:
\[ F_{[2,4]}\ q\ =\ XX(q\ \lor\ Xq\ \lor\ XXq) \]
and $p\ U_{[3,4]}\ q$ can be rewritten as:
\[ p\ U_{[3,4]}\ q\ =\ (p\ \land\ Xp\ \land\  XXp)\ \land\ 
	XXX(q\ \lor\ (p\ \land\ Xq)) \]
The first part of the property specifies that $p$ be must be true in the
present cycle and the next two cycles. The second part of the property
specifies that $q$ must be true in the third cycle, failing which, $p$
must be true in the third cycle and $q$ must be true in the fourth cycle.

\noindent
Therefore, real time operators actually help us to succinctly express
properties that would require too many $X$ operators otherwise.
